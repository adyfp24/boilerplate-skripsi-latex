\chapter{PENDAHULUAN} % Hapus bintang (*) agar bab diberi nomor secara otomatis
% Bab pendahuluan menarasikan tentang langkah awal sebuah penelitian. Bab pendahuluan untuk skripsi terdisi dari beberapa sub-bab yang meliputi latar belakang, rumusan masalah, batasan masalah, tujuan penelitian dan manfaat penelitian. Batasan jumlah halaman untuk Bab Pendahuluan adalah maksimal 4 halaman. Penulisan nomor halaman di awal bab di bagian tengah bawah, untuk halaman selanjutnya berada di pojok kanan atas. [cite: 130, 131, 132] % Ini adalah komentar tentang juknis, bukan konten bab

Bab pendahuluan menarasikan tentang langkah awal sebuah penelitian. Bab pendahuluan untuk skripsi terdiri dari beberapa sub-bab yang meliputi latar belakang, rumusan masalah, batasan masalah, tujuan penelitian dan manfaat penelitian. Batasan jumlah halaman untuk Bab Pendahuluan adalah maksimal 4 halaman. Penulisan nomor halaman di awal bab di bagian tengah bawah, untuk halaman selanjutnya berada di pojok kanan atas.

\section{Latar Belakang} % Ini akan menjadi 1.1 Latar Belakang
Berisi penjelasan berupa alasan penting dan menarik serta perlunya dilakukan penelitian dari masalah yang menjadi topik penelitian. Latar belakang dituliskan secara sistematis, logis dan didasarkan pada alasan/argumentasi ilmiah. Penulisan latar belakang berupa paragraf, di mana satu paragraf berisi satu ide pokok dan terdiri dari minimal 2 kalimat.

Menurut penelitian~\cite{sihombing2021comparison}, Lorem ipsum dolor sit amet, consectetur adipiscing elit. Nullam auctor, nisl eget ultricies tincidunt, nisl nisl aliquam nisl, eget ultricies nisl nisl eget nisl. Lorem ipsum dolor sit amet, consectetur adipiscing elit. Nullam auctor, nisl eget ultricies tincidunt, nisl nisl aliquam nisl, eget ultricies nisl nisl eget nisl.

\section{Rumusan Masalah} % Ini akan menjadi 1.2 Rumusan Masalah
Berisi thesis statement atau research question yang ditulis secara singkat, padat, dan sistematis tentang permasalahan yang diteliti.

\section{Batasan Penelitian (Jika Diperlukan)} % Ini akan menjadi 1.3 Batasan Penelitian
Menerangkan tentang ruang lingkup masalah yang ingin dibatasi oleh peneliti yang disebabkan masalah yang terlalu luas atau lebar yang bisa mengakibatkan penelitian itu tidak bisa fokus. Contoh batasan penelitian antara lain: lokasi, metode, data, asumsi. 

\section{Tujuan Penelitian} % Ini akan menjadi 1.4 Tujuan Penelitian
Menerangkan tentang ruang lingkup masalah yang ingin dibatasi oleh peneliti yang disebabkan masalah yang terlalu luas atau lebar yang bisa mengakibatkan penelitian itu tidak bisa fokus. Contoh batasan penelitian antara lain: lokasi, metode, data, asumsi.

\section{Manfaat Penelitian (Jika Diperlukan)} % Ini akan menjadi 1.5 Manfaat Penelitian
Manfaat penelitian berisi uraian faedah yang diharapkan dari tugas akhir, baik dari sisi pengembangan ilmu pengetahuan maupun dari sisi penerapannya. Manfaat dinyatakan secara operasional dalam kata kerja yang logis.

\section{Keaslian Penelitian (Jika Diperlukan)} % Ini akan menjadi 1.6 Keaslian Penelitian
Berisi uraian yang menunjukkan perbedaan ataupun perbaikan terhadap penelitian terdahulu. Tugas akhir berupa tesis dan disertasi wajib menunjukkan kebaruan berupa konsep, metode, ilmu, dan teknologi, sehingga memenuhi syarat untuk dipublikasikan di jurnal nasional terindeks Sinta atau jurnal internasional bereputasi.