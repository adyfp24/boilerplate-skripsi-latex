\documentclass[12pt, a4paper, onecolumn, oneside]{report}
\usepackage[indonesian,provide=*]{babel}
\usepackage[T1]{fontenc}
\usepackage[utf8]{inputenc}
\usepackage{setspace}
\usepackage{graphicx}
\usepackage{titlesec}
\usepackage{hyperref}
\usepackage[authoryear]{natbib}

\onehalfspacing

% Format halaman
\usepackage[top=3cm, bottom=3cm, left=4cm, right=3cm]{geometry}
\usepackage{fancyhdr}
\usepackage{indentfirst}

% Setup Header
\pagestyle{fancy}
\fancyhf{}
\fancyhead[R]{\thepage}
\fancyhead[L]{\nouppercase{\leftmark}}

% Format judul bab
\titleformat{\chapter}[display]
{\normalfont\bfseries\centering}
{\chaptertitlename\ \thechapter}{20pt}{\Huge}

% ==============================================
% DATA IDENTITAS SKRIPSI (EDIT INI)
\newcommand{\judul}{LOREM IPSUM DOLOR SIT AMET CONSECTETUR ADIPISCING ELIT SED DO EIUSMOD TEMPOR INCIDIDUNT UT LABORE}
\newcommand{\maksud}{Diajukan untuk Memenuhi Sebagian Persyaratan\\Memperoleh Gelar Sarjana}
\newcommand{\jeniskarya}{SKRIPSI}
\newcommand{\nama}{NAMA LENGKAP ANDA}
\newcommand{\nim}{NIM ANDA}
\newcommand{\prodi}{PROGRAM STUDI ANDA}
\newcommand{\fakultas}{FAKULTAS ANDA}
\newcommand{\tahunterbit}{TAHUN}
% ==============================================

\begin{document}

% ================= HALAMAN SAMPUL =================
\begin{titlepage}
    \centering
    \vspace*{1cm}
    
    % Logo UNEJ
    \includegraphics[width=5cm]{resources/images/logo_unej.png}\par
    \vspace{1cm}
    
    % Judul
    \textbf{\Large \judul} \par
    \vspace{1.5cm}
    
    % Maksud Penelitian
    \textbf{\maksud} \par
    \vspace{1cm}
    
    % Jenis Karya Ilmiah
    \textbf{\jeniskarya} \par
    \vspace{1.5cm}
    
    % Identitas Mahasiswa
    \textbf{\nama} \\
    \textbf{\nim} \par
    \vspace{1.5cm}
    
    % Institusi
    \textbf{KEMENTERIAN PENDIDIKAN, KEBUDAYAAN, RISET, DAN TEKNOLOGI} \\
    \textbf{UNIVERSITAS JEMBER} \\
    \textbf{\fakultas} \\
    \textbf{\prodi} \\
    \textbf{\tahunterbit}
\end{titlepage}

% ================= HALAMAN JUDUL =================
\cleardoublepage
\begin{titlepage}
    \centering
    \vspace*{1cm}
    
    \textbf{\Large \judul} \par
    \vspace{1.5cm}
    
    \textbf{\maksud} \par
    \vspace{1cm}
    
    \textbf{\jeniskarya} \par
    \vspace{1.5cm}
    
    Disusun oleh: \\
    \textbf{\nama} \\
    \textbf{\nim} \par
    \vspace{1.5cm}
    
    \textbf{\fakultas} \\
    \textbf{\prodi} \\
    \textbf{UNIVERSITAS JEMBER} \\
    \textbf{\tahunterbit}
\end{titlepage}

% ================= HALAMAN PERSEMBAHAN =================
\cleardoublepage
\chapter*{HALAMAN PERSEMBAHAN}
\addcontentsline{toc}{chapter}{HALAMAN PERSEMBAHAN}
Isi halaman persembahan...

% ================= HALAMAN PERNYATAAN ORISINALITAS =================
\cleardoublepage
\chapter*{PERNYATAAN ORISINALITAS}
\addcontentsline{toc}{chapter}{PERNYATAAN ORISINALITAS}
\begin{center}
    \textbf{\underline{SURAT PERNYATAAN}} \\
    \textbf{KEASLIAN TULISAN}
\end{center}

\vspace{1cm}

Yang bertanda tangan di bawah ini:

Nama : \nama \\
NIM : \nim \\
Program Studi : \prodi \\
Fakultas : \fakultas \\

Dengan ini menyatakan bahwa karya ilmiah yang berjudul:

\judul

adalah benar-benar hasil karya sendiri dan bukan plagiasi...

% ================= HALAMAN PERSETUJUAN =================
\cleardoublepage
\chapter*{HALAMAN PERSETUJUAN}
\addcontentsline{toc}{chapter}{HALAMAN PERSETUJUAN}
Isi halaman persetujuan...

% ================= ABSTRAK =================
\cleardoublepage
\chapter*{ABSTRAK}
\addcontentsline{toc}{chapter}{ABSTRAK}
Isi abstrak dalam Bahasa Indonesia...

\vspace{1cm}
\textbf{Kata kunci}: Kata1, Kata2, Kata3

% ================= ABSTRACT =================
\cleardoublepage
\chapter*{ABSTRACT}
\addcontentsline{toc}{chapter}{ABSTRACT}
English abstract here...

\vspace{1cm}
\textbf{Keywords}: Word1, Word2, Word3

% ================= DAFTAR ISI & GAMBAR =================
\cleardoublepage
\tableofcontents
\cleardoublepage
\listoffigures
\cleardoublepage
\listoftables

% ================= BAB-BAB SKRIPSI =================
\cleardoublepage
\chapter{PENDAHULUAN}

\section{Latar Belakang}
Lorem ipsum dolor sit amet, consectetur adipiscing elit. Nullam auctor, nisl eget ultricies tincidunt, nisl nisl aliquam nisl, eget ultricies nisl nisl eget nisl. Nullam auctor, nisl eget ultricies tincidunt, nisl nisl aliquam nisl, eget ultricies nisl nisl eget nisl (\textit{sea level rise}). 

Lorem ipsum dolor sit amet, consectetur adipiscing elit. Nullam auctor, nisl eget ultricies tincidunt, nisl nisl aliquam nisl, eget ultricies nisl nisl eget nisl. Lorem ipsum dolor sit amet, consectetur adipiscing elit. Nullam auctor, nisl eget ultricies tincidunt, nisl nisl aliquam nisl, eget ultricies nisl nisl eget nisl (\textit{Sea Level Anomaly}).

Lorem ipsum dolor sit amet, consectetur adipiscing elit. Nullam auctor, nisl eget ultricies tincidunt, nisl nisl aliquam nisl, eget ultricies nisl nisl eget nisl (\textit{Long Short-Term Memory}). Lorem ipsum dolor sit amet, consectetur adipiscing elit. Nullam auctor, nisl eget ultricies tincidunt, nisl nisl aliquam nisl, eget ultricies nisl nisl eget nisl (GIS).

\section{Rumusan Masalah}
Berdasarkan latar belakang di atas, rumusan masalah dalam penelitian ini adalah:
\begin{enumerate}
    \item Lorem ipsum dolor sit amet, consectetur adipiscing elit?
    \item Nullam auctor, nisl eget ultricies tincidunt, nisl nisl aliquam nisl?
    \item Eget ultricies nisl nisl eget nisl, consectetur adipiscing elit?
\end{enumerate}

\section{Batasan Penelitian}
Penelitian ini memiliki batasan sebagai berikut:
\begin{itemize}
    \item Lorem ipsum dolor sit amet, consectetur adipiscing elit
    \item Nullam auctor, nisl eget ultricies tincidunt
    \item Nisl nisl aliquam nisl, eget ultricies nisl
    \item Eget ultricies nisl nisl eget nisl
    \item Consectetur adipiscing elit
\end{itemize}

\section{Tujuan Penelitian}
Adapun tujuan dari penelitian ini adalah:
\begin{itemize}
    \item Lorem ipsum dolor sit amet, consectetur adipiscing elit
    \item Nullam auctor, nisl eget ultricies tincidunt
    \item Nisl nisl aliquam nisl, eget ultricies nisl
\end{itemize}

\section{Manfaat Penelitian}
Penelitian ini diharapkan dapat memberikan manfaat sebagai berikut:
\begin{itemize}
    \item Lorem ipsum dolor sit amet, consectetur adipiscing elit
    \item Nullam auctor, nisl eget ultricies tincidunt
    \item Nisl nisl aliquam nisl, eget ultricies nisl
\end{itemize}
\cleardoublepage
\input{chapters/bab2}
\cleardoublepage
\input{chapters/bab3}
\cleardoublepage
\input{chapters/bab4}
\cleardoublepage
\input{chapters/bab5}

% ================= DAFTAR PUSTAKA =================
\cleardoublepage
\renewcommand{\bibname}{DAFTAR PUSTAKA}
\addcontentsline{toc}{chapter}{DAFTAR PUSTAKA}
\bibliographystyle{apalike}  % Atau style lain sesuai ketentuan
\bibliography{references}


% ================= LAMPIRAN (JIKA ADA) =================
\cleardoublepage
\appendix
\chapter*{LAMPIRAN}
\addcontentsline{toc}{chapter}{LAMPIRAN}
Isi lampiran...

\end{document}