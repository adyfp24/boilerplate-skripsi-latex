\documentclass[12pt, a4paper, onecolumn, oneside]{report}
\usepackage[indonesian,provide=*]{babel}
\usepackage[T1]{fontenc}
\usepackage[utf8]{inputenc}
\usepackage{setspace}
\usepackage{graphicx}
\usepackage{titlesec}
\usepackage[colorlinks=true,
            linkcolor=black,
            citecolor=black,
            urlcolor=blue]{hyperref}
% \usepackage[authoryear]{natbib}
\usepackage[authoryear]{natbib}

\onehalfspacing

% Format halaman
\usepackage[top=3cm, bottom=3cm, left=4cm, right=3cm]{geometry}
\usepackage{fancyhdr}
\usepackage{indentfirst}
\usepackage{enumitem}

% Setup Header
\pagestyle{fancy}
\fancyhf{}
\fancyhead[R]{\thepage}
\fancyhead[L]{\nouppercase{\leftmark}}

% Format judul bab
\titleformat{\chapter}[display]
  {\normalfont\bfseries\centering} 
  {\MakeUppercase{\chaptertitlename\ \thechapter.}}
  {1em}
  {\fontsize{12pt}{14pt}\selectfont\MakeUppercase}

% Format subjudul bab
\titleformat{\section}
  {\normalfont\bfseries\fontsize{12pt}{14pt}\selectfont} 
  {\thesection} 
  {1em} 
  {}

%   Format jarak antar judul bab
\titlespacing*{\section}
  {0pt}{1.5ex plus .1ex minus .2ex}{1ex plus .1ex}



% ==============================================
% DATA IDENTITAS SKRIPSI (EDIT INI)
\newcommand{\thetitle}{JUDUL SKRIPSI ANDA DI SINI} % Berbentuk frasa, tidak diawali dengan kata kerja, tidak lebih dari 15 kata 
\newcommand{\theauthor}{Mawar Indah Berseri} % 
\newcommand{\thenim}{200810201008} % 
\newcommand{\thedegree}{\textbf{Sarjana}} % Sesuaikan dengan gelar yang diperoleh 
\newcommand{\thestudyprogram}{\textless\textless NAMA PROGRAM STUDI\textgreater\textgreater} % [cite: 3, 11]
\newcommand{\thetypeofreport}{SKRIPSI} % Sesuaikan: LAPORAN PRAKTIK KERJA LAPANGAN/ SKRIPSI/ TESIS/ DISERTASI 
\newcommand{\thefaculty}{\textless\textless NAMA FAKULTAS\textgreater\textgreater} % [cite: 10]
\newcommand{\theyear}{\textless\textless TAHUN\textgreater\textgreater} %
% ==============================================

\begin{document}

% ================= HALAMAN SAMPUL =================
\begin{titlepage}
    \centering
    \vspace*{2cm} % Menyesuaikan sedikit jarak atas, bisa disesuaikan lagi

    % Logo UNEJ
    \includegraphics[width=0.3\textwidth]{resources/images/logo_unej.png}\par % Menggunakan width relatif terhadap lebar teks
    \vspace{1cm}

    % Judul [cite: 1, 2]
    {\bfseries\fontsize{14pt}{16pt}\selectfont \MakeUppercase{\thetitle}} % Bold 14pt, Huruf kapital, Spasi 1 (14pt/16pt) 
    \vspace{1.5cm}

    % Maksud Penelitian 
    diajukan untuk memenuhi sebagian persyaratan memperoleh gelar \\\thedegree\ pada program studi \thestudyprogram. % Frasa sesuai dokumen 
    \vspace{1cm}

    % Jenis Karya Ilmiah 
    \textbf{\MakeUppercase{\thetypeofreport}} % Huruf kapital dan bold 
    \vspace{1.5cm}

    % Oleh 
    Oleh \par % Ditambahkan "Oleh" 
    \vspace{0.5cm}

    % Identitas Mahasiswa [cite: 6, 7]
    \textbf{\theauthor} \\ % Nama 
    \textbf{\thenim} \par % NIM 
    \vspace{2cm} % Jarak setelah identitas mahasiswa

    % Institusi
    \textbf{KEMENTERIAN PENDIDIKAN, KEBUDAYAAN, RISET, DAN TEKNOLOGI} \\
    \textbf{UNIVERSITAS JEMBER} \\
    \textbf{\MakeUppercase{\thefaculty}} \\ 
    \textbf{\MakeUppercase{\thestudyprogram}} \\ 
    \textbf{JEMBER} \\
    \textbf{\theyear} % 
\end{titlepage}

% ================= HALAMAN JUDUL =================
% \cleardoublepage
% \begin{titlepage}
%     \centering
%     \vspace*{1cm}
    
%     \textbf{\Large \judul} \par
%     \vspace{1.5cm}
    
%     \textbf{\maksud} \par
%     \vspace{1cm}
    
%     \textbf{\jeniskarya} \par
%     \vspace{1.5cm}
    
%     Disusun oleh: \\
%     \textbf{\nama} \\
%     \textbf{\nim} \par
%     \vspace{1.5cm}
    
%     \textbf{\fakultas} \\
%     \textbf{\prodi} \\
%     \textbf{UNIVERSITAS JEMBER} \\
%     \textbf{\tahunterbit}
% \end{titlepage}

% ================= HALAMAN PERSEMBAHAN =================
\cleardoublepage
\chapter*{PERSEMBAHAN}
\addcontentsline{toc}{chapter}{PERSEMBAHAN}
Halaman persembahan berisi ucapan kepada siapa tugas akhir dipersembahkan dengan menggunakan bahasa baku dan tidak terkesan berlebihan.

% ================= HALAMAN MOTTO =================
\chapter*{MOTTO}
\addcontentsline{toc}{chapter}{MOTTO}
Halaman untuk menuliskan penyemangat mahasiswa dalam penyelesaian skripsi. Motto bisa dikarang sendiri oleh penulis, mengutip dari kitab suci, atau kata-kata dari tokoh yang menginspirasi.


% ================= HALAMAN PERNYATAAN ORISINALITAS =================
\cleardoublepage
\chapter*{PERNYATAAN ORISINALITAS}
\addcontentsline{toc}{chapter}{PERNYATAAN ORISINALITAS}

% Menghapus "SURAT PERNYATAAN KEASLIAN TULISAN" karena tidak ada di gambar baru
% \begin{center}
%     \textbf{\underline{SURAT PERNYATAAN}} \\
%     \textbf{KEASLIAN TULISAN}
% \end{center}

% Mengatur ulang spasi baris ke normal untuk bagian ini jika sebelumnya diatur 1.5
% (Jika Anda memiliki \onehalfspacing global, Anda mungkin perlu mengatur ulang secara lokal)
% \singlespacing % Opsional, jika Anda ingin bagian ini spasi tunggal dan sisanya beda

\vspace{0.5cm} % Spasi dari atas halaman

\noindent Saya yang bertanda tangan di bawah ini :
\begin{itemize}
    \item[] Nama \quad\quad: ................................................
    \item[] NIM \quad\quad\quad: ................................................
\end{itemize}
% Menghapus Program Studi dan Fakultas karena tidak ada di gambar
% Program Studi : \thestudyprogram \\
% Fakultas : \thefaculty \\

\noindent Menyatakan dengan sesungguhnya bahwa skripsi yang berjudul: \textit{\thetitle} (tulisan judul menggunakan huruf miring title case)
adalah benar-benar hasil karya sendiri, kecuali jika dalam pengutipan substansi disebutkan sumbernya, dan belum pernah diajukan pada institusi manapun, serta bukan karya jiplakan.
Saya bertanggung jawab atas keabsahan dan kebenaran isinya sesuai dengan sikap ilmiah yang harus dijunjung tinggi.
Demikian pernyataan ini saya buat dengan sebenarnya, tanpa adanya tekanan dan paksaan dari pihak manapun serta bersedia mendapat sanksi akademik jika ternyata di kemudian hari pernyataan ini tidak benar.

\vspace{1.5cm} % Spasi sebelum tanggal dan tanda tangan

\raggedleft Jember, (tanggal pernyataan dibuat)

\raggedleft Yang menyatakan, \\
\vspace{1cm} % Spasi untuk tanda tangan
\raggedleft (tanda tangan) \\
\raggedleft (Meterai Rp 10.000,00)

\vspace{1cm} % Spasi antara meterai dan nama
\raggedleft (nama) \\
\raggedleft NIM ..........................
\raggedright
\clearpage
% ================= HALAMAN PERSETUJUAN =================
\cleardoublepage
\chapter*{HALAMAN PERSETUJUAN}
\addcontentsline{toc}{chapter}{HALAMAN PERSETUJUAN}

\noindent Skripsi berjudul \textit{\thetitle} (cetak miring) telah diuji dan disetujui oleh \\
\noindent Fakultas \underline{\hspace{4cm}} Universitas Jember pada: \\
\noindent Hari \quad\quad\quad: Senin \\
\noindent Tanggal \quad\quad: \underline{\hspace{0.8cm}} Februari 20XX \\
\noindent Tempat \quad\quad: Fakultas \underline{\hspace{4cm}} Universitas Jember \\

\vspace{0.5cm}
\noindent Pembimbing \hfill Tanda Tangan \\

\vspace{0.5cm} % Spasi sebelum Pembimbing Utama

\noindent 1. Pembimbing Utama \\
Nama \quad: \underline{\hspace{5cm}} \hfill (…………………….) \\
NIP \quad\quad: \underline{\hspace{5cm}} \\

\vspace{0.5cm} % Spasi sebelum Pembimbing Anggota (jika diperlukan)

\noindent 2. Pembimbing Anggota (\textit{jika diperlukan}) \\
Nama \quad: \underline{\hspace{5cm}} \hfill (…………………….) \\
NIP \quad\quad: \underline{\hspace{5cm}} \\

\vspace{1cm} % Spasi sebelum Penguji

\noindent Penguji \\

\vspace{0.5cm} % Spasi sebelum Penguji Utama

\noindent 1. Penguji Utama \\
Nama \quad: \underline{\hspace{5cm}} \hfill (…………………….) \\
NIP \quad\quad: \underline{\hspace{5cm}} \\

\vspace{0.5cm} % Spasi sebelum Penguji Anggota 1

\noindent 2. Penguji Anggota 1 \\
Nama \quad: \underline{\hspace{5cm}} \hfill (…………………….) \\
NIP \quad\quad: \underline{\hspace{5cm}} \\

\vspace{0.5cm} % Spasi sebelum Penguji Anggota 2 (jika diperlukan)

\noindent 3. Penguji Anggota 2 (\textit{jika diperlukan}) \\
Nama \quad: \underline{\hspace{5cm}} \hfill (…………………….) \\
NIP \quad\quad: \underline{\hspace{5cm}} \\

\vspace{1.5cm} % Spasi sebelum Catatan

\noindent \textbf{Catatan:} \\
\begin{enumerate}[label=\alph*)] % Menggunakan enumitem untuk label a) b)
    \item Untuk tesis dan disertasi, jumlah pembimbing dan penguji menyesuaikan kebijakan program studi  \\
    \item Khusus untuk disertasi, halaman pengesahan dapat ditambahkan tanda tangan dekan \\
\end{enumerate}

\clearpage

% ================= ABSTRAK =================
\cleardoublepage
\chapter*{ABSTRAK}
\addcontentsline{toc}{chapter}{ABSTRAK}
Isi abstrak dalam Bahasa Indonesia...

\vspace{1cm}
\textbf{Kata kunci}: Kata1, Kata2, Kata3

% ================= ABSTRACT =================
\cleardoublepage
\chapter*{ABSTRACT}
\addcontentsline{toc}{chapter}{ABSTRACT}
English abstract here...

\vspace{1cm}
\textbf{Keywords}: Word1, Word2, Word3

% ================= DAFTAR ISI & GAMBAR =================
\cleardoublepage
\tableofcontents
\cleardoublepage
\listoffigures
\cleardoublepage
\listoftables

% ================= BAB-BAB SKRIPSI =================
% BAB 1. PENDAHULUAN ================================
\cleardoublepage
\addcontentsline{toc}{chapter}{Bab 1. PENDAHULUAN}
\chapter{PENDAHULUAN}

\section{Latar Belakang}
Lorem ipsum dolor sit amet, consectetur adipiscing elit. Nullam auctor, nisl eget ultricies tincidunt, nisl nisl aliquam nisl, eget ultricies nisl nisl eget nisl. Nullam auctor, nisl eget ultricies tincidunt, nisl nisl aliquam nisl, eget ultricies nisl nisl eget nisl (\textit{sea level rise}). 

Lorem ipsum dolor sit amet, consectetur adipiscing elit. Nullam auctor, nisl eget ultricies tincidunt, nisl nisl aliquam nisl, eget ultricies nisl nisl eget nisl. Lorem ipsum dolor sit amet, consectetur adipiscing elit. Nullam auctor, nisl eget ultricies tincidunt, nisl nisl aliquam nisl, eget ultricies nisl nisl eget nisl (\textit{Sea Level Anomaly}).

Lorem ipsum dolor sit amet, consectetur adipiscing elit. Nullam auctor, nisl eget ultricies tincidunt, nisl nisl aliquam nisl, eget ultricies nisl nisl eget nisl (\textit{Long Short-Term Memory}). Lorem ipsum dolor sit amet, consectetur adipiscing elit. Nullam auctor, nisl eget ultricies tincidunt, nisl nisl aliquam nisl, eget ultricies nisl nisl eget nisl (GIS).

\section{Rumusan Masalah}
Berdasarkan latar belakang di atas, rumusan masalah dalam penelitian ini adalah:
\begin{enumerate}
    \item Lorem ipsum dolor sit amet, consectetur adipiscing elit?
    \item Nullam auctor, nisl eget ultricies tincidunt, nisl nisl aliquam nisl?
    \item Eget ultricies nisl nisl eget nisl, consectetur adipiscing elit?
\end{enumerate}

\section{Batasan Penelitian}
Penelitian ini memiliki batasan sebagai berikut:
\begin{itemize}
    \item Lorem ipsum dolor sit amet, consectetur adipiscing elit
    \item Nullam auctor, nisl eget ultricies tincidunt
    \item Nisl nisl aliquam nisl, eget ultricies nisl
    \item Eget ultricies nisl nisl eget nisl
    \item Consectetur adipiscing elit
\end{itemize}

\section{Tujuan Penelitian}
Adapun tujuan dari penelitian ini adalah:
\begin{itemize}
    \item Lorem ipsum dolor sit amet, consectetur adipiscing elit
    \item Nullam auctor, nisl eget ultricies tincidunt
    \item Nisl nisl aliquam nisl, eget ultricies nisl
\end{itemize}

\section{Manfaat Penelitian}
Penelitian ini diharapkan dapat memberikan manfaat sebagai berikut:
\begin{itemize}
    \item Lorem ipsum dolor sit amet, consectetur adipiscing elit
    \item Nullam auctor, nisl eget ultricies tincidunt
    \item Nisl nisl aliquam nisl, eget ultricies nisl
\end{itemize}

\cleardoublepage
\addcontentsline{toc}{chapter}{Bab 2. TINJAUAN TEORI}
\input{chapters/bab2}
\cleardoublepage

\addcontentsline{toc}{chapter}{Bab 3. METODOLOGI PENELITIAN}
\input{chapters/bab3}
\cleardoublepage
\addcontentsline{toc}{chapter}{Bab 4. HASIL DAN PEMBAHASAN}
\input{chapters/bab4}
\cleardoublepage
\addcontentsline{toc}{chapter}{Bab 5. KESIMPULAN DAN SARAN}
\input{chapters/bab5}

% ================= DAFTAR PUSTAKA =================
\cleardoublepage
\renewcommand{\bibname}{DAFTAR PUSTAKA}
\addcontentsline{toc}{chapter}{DAFTAR PUSTAKA}
\bibliographystyle{apalike}
\bibliography{references}    


% ================= LAMPIRAN (JIKA ADA) =================
\cleardoublepage
\appendix
\chapter*{LAMPIRAN}
\addcontentsline{toc}{chapter}{LAMPIRAN}
Isi lampiran...

\end{document}