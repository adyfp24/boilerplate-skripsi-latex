\documentclass[12pt, a4paper, onecolumn, oneside]{report}
\usepackage[indonesian,provide=*]{babel}
\usepackage[T1]{fontenc}
\usepackage[utf8]{inputenc}
\usepackage{setspace}

\onehalfspacing

% Paket tambahan lainnya tetap sama...

% Paket tambahan untuk format akademik
\usepackage{indentfirst}
\usepackage[top=3cm, bottom=3cm, left=4cm, right=3cm]{geometry}
\usepackage{fancyhdr}
% \usepackage{enumitem}

% Setup Header
\pagestyle{fancy}
\fancyhf{}
\fancyhead[R]{\thepage}
\fancyhead[L]{\nouppercase{\leftmark}}

\title{Pengembangan model prediksi kenaikan muka air laut (sea level rise) berdasarkan data Sea Level Anomaly menggunakan Long Short-Term Memory serta pemetaan wilayah terdampak menggunakan teknologi GIS di kawasan pesisir Jember.}
\author{NAMA ANDA \\ NIM ANDA}
\date{\today}

\begin{document}

\maketitle
\tableofcontents
\listoffigures  % Opsional jika ada gambar
\listoftables   % Opsional jika ada tabel

% Memanggil file bab
\chapter{PENDAHULUAN}

\section{Latar Belakang}
Lorem ipsum dolor sit amet, consectetur adipiscing elit. Nullam auctor, nisl eget ultricies tincidunt, nisl nisl aliquam nisl, eget ultricies nisl nisl eget nisl. Nullam auctor, nisl eget ultricies tincidunt, nisl nisl aliquam nisl, eget ultricies nisl nisl eget nisl. 

Lorem ipsum dolor sit amet, consectetur adipiscing elit. Nullam auctor, nisl eget ultricies tincidunt, nisl nisl aliquam nisl, eget ultricies nisl nisl eget nisl. Lorem ipsum dolor sit amet, consectetur adipiscing elit. Nullam auctor, nisl eget ultricies tincidunt, nisl nisl aliquam nisl, eget ultricies nisl nisl eget nisl.

Lorem ipsum dolor sit amet, consectetur adipiscing elit. Nullam auctor, nisl eget ultricies tincidunt, nisl nisl aliquam nisl, eget ultricies nisl nisl eget nisl. Lorem ipsum dolor sit amet, consectetur adipiscing elit. Nullam auctor, nisl eget ultricies tincidunt, nisl nisl aliquam nisl, eget ultricies nisl nisl eget nisl.

Menurut penelitian \citet{sihombing2021comparison}, Lorem ipsum dolor sit amet, consectetur adipiscing elit. Nullam auctor, nisl eget ultricies tincidunt, nisl nisl aliquam nisl, eget ultricies nisl nisl eget nisl. Lorem ipsum dolor sit amet, consectetur adipiscing elit. Nullam auctor, nisl eget ultricies tincidunt, nisl nisl aliquam nisl, eget ultricies nisl nisl eget nisl.

\section{Rumusan Masalah}
Berdasarkan latar belakang di atas, rumusan masalah dalam penelitian ini adalah:
\begin{enumerate}
    \item Lorem ipsum dolor sit amet, consectetur adipiscing elit?
    \item Nullam auctor, nisl eget ultricies tincidunt, nisl nisl aliquam nisl?
    \item Eget ultricies nisl nisl eget nisl, consectetur adipiscing elit?
\end{enumerate}

\section{Batasan Penelitian}
Penelitian ini memiliki batasan sebagai berikut:
\begin{itemize}
    \item Lorem ipsum dolor sit amet, consectetur adipiscing elit
    \item Nullam auctor, nisl eget ultricies tincidunt
    \item Nisl nisl aliquam nisl, eget ultricies nisl
    \item Eget ultricies nisl nisl eget nisl
    \item Consectetur adipiscing elit
\end{itemize}

\section{Tujuan Penelitian}
Adapun tujuan dari penelitian ini adalah:
\begin{itemize}
    \item Lorem ipsum dolor sit amet, consectetur adipiscing elit
    \item Nullam auctor, nisl eget ultricies tincidunt
    \item Nisl nisl aliquam nisl, eget ultricies nisl
\end{itemize}

\section{Manfaat Penelitian}
Penelitian ini diharapkan dapat memberikan manfaat sebagai berikut:
\begin{itemize}
    \item Lorem ipsum dolor sit amet, consectetur adipiscing elit
    \item Nullam auctor, nisl eget ultricies tincidunt
    \item Nisl nisl aliquam nisl, eget ultricies nisl
\end{itemize}

% Contoh memanggil bab lain (buat file terpisah)
% \input{chapters/bab2} 
% \input{chapters/bab3}

% Daftar Pustaka (sesuaikan dengan style yang diwajibkan)
\bibliographystyle{unsrt}  % atau style lain seperti ieeetr
\bibliography{references}  % file referensi.bib

\end{document}